%% Path B — Fusion Engineering and Design (Elsevier)
%% "Stochastic Neural Control of Tokamak Vertical Position
%%  via Compiled Petri Nets"
%%
%% Compile: pdflatex -> bibtex -> pdflatex x2  (see Makefile)
%% -----------------------------------------------------------
\documentclass[preprint,12pt,authoryear]{elsarticle}

\usepackage{amsmath,amssymb,amsfonts}
\usepackage{graphicx}
\usepackage{hyperref}
\usepackage{booktabs}
\usepackage{siunitx}
\usepackage{algorithm}
\usepackage{algpseudocode}
\usepackage{subcaption}
\usepackage{xcolor}
\usepackage{tikz}
\usetikzlibrary{petri,arrows.meta,positioning}

% Placeholder text (remove for final submission)
\usepackage{lipsum}

\journal{Fusion Engineering and Design}

\begin{document}

\begin{frontmatter}

\title{Stochastic Neural Control of Tokamak Vertical Position\\
       via Compiled Petri Nets}

\author[anulum]{Miroslav Sotek\corref{cor1}}
\ead{miroslav.sotek@anulum.research}
\author[anulum]{Michal Reiprich}

\cortext[cor1]{Corresponding author.}

\affiliation[anulum]{organization={Anulum Research},
                     country={Czech Republic}}

\begin{abstract}
% TODO: Write abstract (~200 words)
% Key points to cover:
% - Vertical position control is safety-critical in tokamaks
% - Propose a formally verifiable controller based on Petri nets
% - Petri net is compiled into a spiking neural network (SNN)
% - Stochastic firing semantics provide noise robustness
% - Formal properties (boundedness, liveness, mutual exclusion) preserved
% - Benchmarked against PID baseline on DIII-D-like vertical instability model
% - Comparable or superior disturbance rejection with formal guarantees
\lipsum[1]
\end{abstract}

\begin{keyword}
tokamak \sep vertical stability \sep Petri net \sep spiking neural network \sep
stochastic computing \sep formal verification \sep plasma control
\end{keyword}

\end{frontmatter}

%% ============================================================
%%  1. INTRODUCTION
%% ============================================================
\section{Introduction}
\label{sec:introduction}

% TODO: Motivate the problem
% Points to cover:
% - Vertical instability is the primary safety concern in elongated tokamaks
% - Loss of vertical control -> VDE -> wall damage, disruptions
% - Current controllers: PID, LQR, model-predictive — none formally verified
% - Formal verification is standard in aerospace/nuclear-fission but absent
%   in fusion plasma control
% - Our contribution: Petri-net-based controller compiled to SNN
%   with provable safety properties

\lipsum[2-3]

Vertical position control in elongated tokamak plasmas is a
safety-critical task: loss of vertical stability leads to vertical
displacement events (VDEs) that can cause severe damage to
plasma-facing components~\cite{ariola2008,humphreys2015}.

% TODO: Discuss the formal verification gap in fusion control
Despite the safety-critical nature of vertical control, existing
controllers in the tokamak community---typically PID or LQR
variants---lack formal guarantees of correctness.  In contrast,
safety-critical systems in aerospace and nuclear fission routinely
employ formally verified controllers.

% TODO: State contributions
In this paper we propose a novel controller architecture in which:
\begin{enumerate}
  \item A Petri net~\cite{petri1962} encodes the control logic with
        formally verifiable structural properties;
  \item The Petri net is compiled into a spiking neural
        network~\cite{maass1997} preserving all verified properties;
  \item Stochastic firing semantics provide inherent noise robustness
        without sacrificing formal guarantees;
  \item The compiled controller is benchmarked against a tuned PID
        baseline on a DIII-D-like vertical instability plant model.
\end{enumerate}

%% ============================================================
%%  2. BACKGROUND
%% ============================================================
\section{Background}
\label{sec:background}

\subsection{Vertical Position Instability}
\label{sec:vertical-instability}

% TODO: Describe the physics of vertical instability
% - Elongated plasmas: n=0 vertical instability growth rate
% - Passive stabilisation via conducting structures (wall, vessel)
% - Active feedback required on timescale < wall time
% - Typical growth rates: ~100-1000 s^-1
% - Controlled by poloidal field coil currents

\lipsum[4]

The vertical instability growth rate $\gamma$ for an elongated plasma
with elongation $\kappa$ scales approximately
as~\cite{ariola2008}:
\begin{equation}
  \gamma \sim \frac{(\kappa^{2}-1)}{\tau_{w}},
  \label{eq:growth-rate}
\end{equation}
where $\tau_w$ is the resistive wall time of the surrounding
conducting structure.

\subsection{Petri Net Formalism}
\label{sec:petri-nets}

% TODO: Define Petri nets formally
% - Definition: bipartite directed graph (P, T, F, W, M_0)
% - Places, transitions, arcs, weights, initial marking
% - Firing rule
% - Key properties: boundedness, liveness, reachability, mutual exclusion
% - Reference: \cite{petri1962,murata1989}

A Petri net~\cite{petri1962} is a bipartite directed graph
$\mathcal{N} = (P, T, F, W, M_0)$ where $P$ is a finite set of
places, $T$ is a finite set of transitions, $F \subseteq (P \times T)
\cup (T \times P)$ is the flow relation, $W: F \to \mathbb{N}^{+}$ is
the weight function, and $M_0: P \to \mathbb{N}$ is the initial
marking~\cite{murata1989}.

\lipsum[5]

\subsection{Spiking Neural Networks}
\label{sec:snn}

% TODO: Describe SNN basics
% - Leaky integrate-and-fire (LIF) neuron model
% - Spike-based communication
% - Temporal coding
% - Computational universality
% - Reference: \cite{maass1997}

Spiking neural networks (SNNs)~\cite{maass1997} represent the
third generation of neural network models, in which information is
encoded in the precise timing of discrete spike events rather than
continuous activation values.

\lipsum[6]

\subsection{Stochastic Computing}
\label{sec:stochastic-computing}

% TODO: Describe stochastic computing principles
% - Values encoded as probabilities in bitstreams
% - Multiplication = AND gate, addition = MUX
% - Inherent noise tolerance
% - Low hardware cost
% - Connection to SC-NeuroCore architecture \cite{scneurocore}

\lipsum[7]

%% ============================================================
%%  3. METHOD
%% ============================================================
\section{Method}
\label{sec:method}

\subsection{Petri Net Controller Design}
\label{sec:pn-design}

% TODO: Describe the Petri net structure for vertical control
% - Places: position_error, integral_error, derivative_error,
%           coil_current_command, safety_interlock
% - Transitions: proportional_action, integral_action,
%                derivative_action, saturation_limit, emergency_stop
% - Guard conditions and inhibitor arcs for safety
% - Figure: Petri net diagram (TikZ)

\lipsum[8]

% TODO: Add TikZ Petri net diagram
\begin{figure}[htbp]
\centering
% \begin{tikzpicture}[node distance=2cm, ...]
%   % TODO: Draw the Petri net
% \end{tikzpicture}
\fbox{\parbox{0.8\textwidth}{\centering\vspace{3cm}
  TODO: Petri net controller diagram\vspace{3cm}}}
\caption{Petri net structure for vertical position control.
  Circles denote places (state variables); bars denote transitions
  (control actions).  Inhibitor arcs enforce mutual exclusion
  between normal operation and emergency shutdown.}
\label{fig:petri-net}
\end{figure}

\subsection{Compilation to Spiking Neural Network}
\label{sec:compilation}

% TODO: Describe the PN -> SNN compilation
% - Each place -> reservoir of LIF neurons
% - Each transition -> coincidence detector neuron
% - Arc weights -> synaptic weights
% - Inhibitor arcs -> inhibitory synapses
% - Token count -> spike rate (rate coding)
% - Compilation algorithm (pseudocode)

\lipsum[9]

\begin{algorithm}[htbp]
\caption{Petri Net to SNN Compilation}
\label{alg:compilation}
\begin{algorithmic}[1]
\Require Petri net $\mathcal{N} = (P, T, F, W, M_0)$
\Ensure SNN graph $\mathcal{S} = (N, E, \mathbf{w})$
\For{each place $p_i \in P$}
  \State Create neuron pool $\mathcal{P}_i$ with $M_0(p_i)$ baseline rate
\EndFor
\For{each transition $t_j \in T$}
  \State Create coincidence neuron $c_j$
  \For{each input arc $(p_i, t_j) \in F$}
    \State Add excitatory synapse $\mathcal{P}_i \to c_j$
          with weight $W(p_i, t_j)$
  \EndFor
  \For{each output arc $(t_j, p_k) \in F$}
    \State Add excitatory synapse $c_j \to \mathcal{P}_k$
          with weight $W(t_j, p_k)$
  \EndFor
\EndFor
\State \Return $\mathcal{S}$
\end{algorithmic}
\end{algorithm}

\subsection{Stochastic Firing Semantics}
\label{sec:stochastic-firing}

% TODO: Describe how stochastic computing is applied
% - Transition firing probability proportional to input marking
% - Stochastic bitstream encoding of continuous control signals
% - Noise injection and its effect on control bandwidth
% - Connection to SC-NeuroCore stochastic compute units

\lipsum[10]

\subsection{Formal Property Preservation}
\label{sec:property-preservation}

% TODO: Prove/argue that compilation preserves formal properties
% - Theorem: If PN is k-bounded, compiled SNN spike rates are bounded
% - Theorem: If PN is live, SNN has no dead neurons
% - Theorem: Mutual exclusion in PN -> inhibitory exclusion in SNN
% - Sketch of proofs or reference to appendix

\lipsum[11]

%% ============================================================
%%  4. FORMAL ANALYSIS
%% ============================================================
\section{Formal Analysis}
\label{sec:formal-analysis}

\subsection{Boundedness}
\label{sec:boundedness}

% TODO: Prove boundedness of the Petri net
% - Coverability tree analysis
% - All places are k-bounded (state k)
% - Physical interpretation: bounded coil currents, bounded errors

\lipsum[12]

\subsection{Liveness}
\label{sec:liveness}

% TODO: Prove liveness
% - Every transition can eventually fire from any reachable marking
% - No deadlocks in the control logic
% - Implication: controller never gets stuck

\lipsum[13]

\subsection{Mutual Exclusion}
\label{sec:mutual-exclusion}

% TODO: Prove mutual exclusion properties
% - Normal operation and emergency shutdown are mutually exclusive
% - Coil current increase and decrease cannot fire simultaneously
% - S-invariant analysis

\lipsum[14]

\subsection{Home State}
\label{sec:home-state}

% TODO: Prove home state property
% - The initial marking M_0 is reachable from every reachable marking
% - Implication: controller can always return to safe idle state
% - T-invariant analysis

\lipsum[15]

%% ============================================================
%%  5. EXPERIMENTAL VALIDATION
%% ============================================================
\section{Experimental Validation}
\label{sec:validation}

\subsection{Plant Model}
\label{sec:plant-model}

% TODO: Describe the vertical instability plant model
% - Linearised rigid body + conducting wall model
% - Parameters from DIII-D: growth rate, wall time, coil response
% - State-space representation
% - Noise model (sensor noise, actuator noise, plasma disturbances)

\lipsum[16]

The vertical dynamics are modelled as a linearised system:
\begin{equation}
  \dot{z}_p = \gamma\, z_p + \frac{1}{\tau_c}\, I_c + d(t),
  \label{eq:plant}
\end{equation}
where $z_p$ is the vertical displacement, $\gamma$ the instability
growth rate, $I_c$ the control coil current, $\tau_c$ the coil
response time, and $d(t)$ represents plasma disturbances.

\subsection{PID Baseline}
\label{sec:pid-baseline}

% TODO: Describe the PID baseline controller
% - Standard PID with anti-windup
% - Tuning method (Ziegler-Nichols or optimization-based)
% - Performance metrics: settling time, overshoot, steady-state error

\lipsum[17]

\subsection{Benchmark Scenarios}
\label{sec:scenarios}

% TODO: Define benchmark scenarios
% - Scenario 1: Step disturbance (e.g., minor disruption)
% - Scenario 2: Ramp disturbance (e.g., beta ramp-up)
% - Scenario 3: Sensor noise rejection
% - Scenario 4: Actuator saturation
% - Scenario 5: Combined worst-case

\lipsum[18]

\subsection{Head-to-Head Results}
\label{sec:results}

% TODO: Present comparative results
% - Table: PID vs PN-SNN for each scenario (settling time, max deviation, etc.)
% - Figures: time traces for each scenario
% - Statistical comparison over Monte Carlo runs

\begin{table}[htbp]
\centering
\caption{Head-to-head comparison: PID vs.\ Petri-net SNN controller.}
\label{tab:results}
\begin{tabular}{@{}lccccc@{}}
\toprule
Metric & PID & PN-SNN & Units & Improvement \\
\midrule
Settling time ($2\%$)     & --- & --- & \si{\ms} & --- \\
Max displacement           & --- & --- & \si{\mm} & --- \\
Steady-state error         & --- & --- & \si{\mm} & --- \\
Noise rejection ($3\sigma$)& --- & --- & \si{\mm} & --- \\
Saturation recovery        & --- & --- & \si{\ms} & --- \\
\bottomrule
\end{tabular}
\end{table}

\lipsum[19]

%% ============================================================
%%  6. DISCUSSION
%% ============================================================
\section{Discussion}
\label{sec:discussion}

% TODO: Discuss results, implications, limitations
% Points to cover:
% - Formal guarantees are the primary advantage
% - Performance comparable to or better than PID
% - Stochastic firing provides inherent robustness
% - Compilation overhead is one-time (offline)
% - Runtime on neuromorphic hardware (SC-NeuroCore) would be sub-microsecond
% - Limitations: linearised plant model, limited disturbance scenarios
% - Future: nonlinear plant, ITER-relevant scenarios, hardware deployment

\lipsum[20-21]

%% ============================================================
%%  7. CONCLUSION
%% ============================================================
\section{Conclusion}
\label{sec:conclusion}

% TODO: Summarise contributions
% - First formally verified controller for tokamak vertical stability
% - Petri net -> SNN compilation preserves all four properties
% - Stochastic firing adds noise robustness without losing guarantees
% - Competitive with PID in simulation benchmarks
% - Path to neuromorphic hardware deployment via SC-NeuroCore

\lipsum[22]

%% ============================================================
%%  8. CODE AVAILABILITY
%% ============================================================
\section{Code Availability}
\label{sec:code-availability}

% TODO: Fill in actual repository URLs
The Petri net controller design, SNN compilation code, and benchmark
scripts are available at:
\begin{itemize}
  \item \textsc{SCPN-Fusion-Core}: \url{https://github.com/TODO/scpn-fusion-core}
  \item \textsc{SC-NeuroCore}: \url{https://github.com/TODO/sc-neurocore}
\end{itemize}
Both repositories are released under the MIT license.

%% ============================================================
%%  ACKNOWLEDGEMENTS
%% ============================================================
\section*{Acknowledgements}

% TODO: Acknowledge funding, collaborators
% - Computational resources
% - DIII-D team for plant model parameters (if applicable)

%% ============================================================
%%  REFERENCES
%% ============================================================
\bibliographystyle{elsarticle-harv}
\bibliography{references}

\end{document}
