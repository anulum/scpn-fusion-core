%% Path A — Computer Physics Communications (Elsevier)
%% "Open-Source Equilibrium Reconstruction for Tokamak Plasmas:
%%  Validation Against DIII-D Discharges"
%%
%% Compile: pdflatex -> bibtex -> pdflatex x2  (see Makefile)
%% -----------------------------------------------------------
\documentclass[preprint,12pt,authoryear]{elsarticle}

\usepackage{amsmath,amssymb,amsfonts}
\usepackage{graphicx}
\usepackage{hyperref}
\usepackage{booktabs}
\usepackage{siunitx}
\usepackage{algorithm}
\usepackage{algpseudocode}
\usepackage{subcaption}
\usepackage{xcolor}

% Placeholder text (remove for final submission)
\usepackage{lipsum}

\journal{Computer Physics Communications}

\begin{document}

\begin{frontmatter}

\title{Open-Source Equilibrium Reconstruction for Tokamak Plasmas:\\
       Validation Against DIII-D Discharges}

\author[anulum]{Miroslav Sotek\corref{cor1}}
\ead{miroslav.sotek@anulum.research}
\author[anulum]{Michal Reiprich}

\cortext[cor1]{Corresponding author.}

\affiliation[anulum]{organization={Anulum Research},
                     country={Czech Republic}}

\begin{abstract}
% TODO: Write abstract (~200 words)
% Key points to cover:
% - Open-source Grad-Shafranov equilibrium solver
% - Picard iteration with SOR for forward solve
% - Levenberg-Marquardt inverse with mtanh parameterisation
% - Tikhonov + Huber regularisation for robustness
% - Rust core with Python bindings (PyO3)
% - Validation against DIII-D experimental discharges
% - Performance benchmarks vs existing codes (EFIT, FREEGS)
\lipsum[1]
\end{abstract}

\begin{keyword}
tokamak \sep equilibrium reconstruction \sep Grad-Shafranov \sep
Levenberg-Marquardt \sep open-source \sep Rust \sep PyO3
\end{keyword}

\end{frontmatter}

%% ============================================================
%%  1. INTRODUCTION
%% ============================================================
\section{Introduction}
\label{sec:introduction}

% TODO: Motivate the need for open-source equilibrium reconstruction codes.
% Points to cover:
% - Equilibrium reconstruction is foundational for tokamak operation
% - EFIT is the de-facto standard but proprietary / export-controlled
% - FREEGS exists but limited inverse capability
% - TORAX (DeepMind) focuses on transport, not equilibrium
% - Gap: no open-source code with full inverse + modern regularisation
% - Our contribution: Rust-based solver with Python API, validated on DIII-D

\lipsum[2-3]

The Grad-Shafranov (GS) equation~\cite{grad1958,shafranov1966} governs
axisymmetric magnetohydrostatic equilibria in tokamak plasmas.
Accurate and rapid solution of the GS equation, both in the forward
(given profiles, compute flux) and inverse (given measurements,
reconstruct profiles) directions, is essential for plasma control,
stability analysis, and discharge design.

% TODO: Expand with discussion of EFIT heritage
The EFIT code~\cite{lao1985,lao2005,sabbagh2023} has served as the
workhorse for equilibrium reconstruction across the world tokamak
programme for nearly four decades.  However, its proprietary nature
limits reproducibility and community-driven improvement.

% TODO: Discuss FREEGS and TORAX
Open-source alternatives such as FREEGS~\cite{freegs} and the
recently released TORAX~\cite{torax2024} address parts of this gap,
but neither provides a complete, validated inverse solver with modern
regularisation techniques.

% TODO: State contributions clearly
In this paper we present \textsc{SCPN-Fusion-Core}, an open-source
equilibrium reconstruction code implemented in Rust with Python
bindings via PyO3.  Our main contributions are:
\begin{enumerate}
  \item A Picard-iteration forward solver with successive over-relaxation (SOR);
  \item A Levenberg-Marquardt inverse solver with analytic \texttt{mtanh}
        Jacobians;
  \item Tikhonov and Huber regularisation for robust profile recovery;
  \item Validation against DIII-D experimental data;
  \item Performance benchmarks demonstrating the advantages of the Rust
        implementation.
\end{enumerate}

%% ============================================================
%%  2. NUMERICAL METHODS
%% ============================================================
\section{Numerical Methods}
\label{sec:methods}

\subsection{Grad-Shafranov Formulation}
\label{sec:gs-formulation}

% TODO: Derive or state the GS equation
% \Delta^* \psi = -\mu_0 R J_\phi = -\mu_0 R^2 p'(\psi) - F F'(\psi)
% Boundary conditions, computational domain, normalised flux

The axisymmetric equilibrium of a magnetically confined plasma is
governed by the Grad-Shafranov equation~\cite{grad1958,shafranov1966}:
\begin{equation}
  \Delta^{*}\psi \equiv R\frac{\partial}{\partial R}
    \!\left(\frac{1}{R}\frac{\partial\psi}{\partial R}\right)
    + \frac{\partial^{2}\psi}{\partial Z^{2}}
  = -\mu_{0}R^{2}\,p'(\psi) - F\,F'(\psi),
  \label{eq:gs}
\end{equation}
where $\psi(R,Z)$ is the poloidal magnetic flux function, $p(\psi)$ is
the plasma pressure, $F(\psi)=RB_\phi$ is the poloidal current
function, and $\Delta^{*}$ is the Grad-Shafranov operator.

\lipsum[4]

\subsection{Picard Iteration with Successive Over-Relaxation}
\label{sec:picard-sor}

% TODO: Describe the iterative scheme
% - Discretisation on (R,Z) grid
% - Five-point stencil for \Delta^*
% - Picard iteration: linearise RHS, solve Poisson-like system
% - SOR acceleration with optimal omega
% - Convergence criterion

\lipsum[5]

\subsection{Levenberg-Marquardt Inverse with mtanh Jacobians}
\label{sec:lm-inverse}

% TODO: Describe the inverse problem formulation
% - Measurement vector: magnetic probes, flux loops, MSE, pressure
% - Parameter vector: mtanh profile coefficients
% - LM update: (J^T J + \lambda diag(J^T J)) \delta p = J^T r
% - Analytic Jacobians for mtanh parameterisation
% - Step-size control and convergence

The Levenberg-Marquardt algorithm~\cite{levenberg1944,marquardt1963}
is employed to solve the nonlinear least-squares inverse problem.

\lipsum[6]

\subsection{Tikhonov and Huber Regularisation}
\label{sec:regularisation}

% TODO: Describe regularisation strategy
% - Tikhonov: \|L(p - p_0)\|^2 penalty
% - Huber loss for outlier-robust residuals
% - L-curve or GCV for hyperparameter selection
% - Tikhonov reference: \cite{tikhonov1963}

Tikhonov regularisation~\cite{tikhonov1963} is applied to stabilise
the inverse problem against measurement noise and ill-conditioning.

\lipsum[7]

%% ============================================================
%%  3. IMPLEMENTATION
%% ============================================================
\section{Implementation}
\label{sec:implementation}

\subsection{Rust Core with PyO3 Bindings}
\label{sec:rust-pyo3}

% TODO: Describe the software architecture
% - Rust crate structure (scpn-fusion-rs)
% - PyO3 bindings for Python interoperability
% - ndarray for matrix operations
% - Parallelism via rayon
% - Memory layout and cache efficiency

\lipsum[8]

\subsection{Performance Considerations}
\label{sec:performance}

% TODO: Discuss performance design choices
% - SIMD vectorisation
% - Cache-friendly grid traversal (row-major vs column-major)
% - Sparse vs dense Jacobian storage
% - Parallelism strategy

\lipsum[9]

%% ============================================================
%%  4. VALIDATION
%% ============================================================
\section{Validation}
\label{sec:validation}

\subsection{Data Sources}
\label{sec:data-sources}

% TODO: Describe DIII-D data used
% - Shot numbers and discharge types (L-mode, H-mode, high-beta)
% - Diagnostic channels: magnetic probes, flux loops, MSE, Thomson scattering
% - Reference EFIT reconstructions from MDS+

\lipsum[10]

\subsection{Forward Solve Results}
\label{sec:forward-results}

% TODO: Present forward solve validation
% - Comparison with analytic Solov'ev solutions
% - Grid convergence study
% - Comparison with FREEGS for identical inputs
% - Figures: flux surfaces, safety factor profiles

\lipsum[11]

\subsection{Inverse Reconstruction Results}
\label{sec:inverse-results}

% TODO: Present inverse reconstruction validation
% - Shot-by-shot comparison with EFIT
% - Residual chi-squared statistics
% - Profile comparisons: p(\psi), q(\psi), J_\phi(R,Z)
% - Effect of regularisation on profile smoothness
% - Figures: reconstructed vs EFIT flux surfaces, profile comparisons

\lipsum[12]

%% ============================================================
%%  5. PERFORMANCE BENCHMARKS
%% ============================================================
\section{Performance Benchmarks}
\label{sec:benchmarks}

\subsection{Grid Scaling}
\label{sec:grid-scaling}

% TODO: Present scaling with grid resolution
% - Wall-clock time vs N_R x N_Z for forward solve
% - Scaling exponent and comparison with O(N^2) expectation
% - Memory usage

\lipsum[13]

\subsection{Solver Comparison}
\label{sec:solver-comparison}

% TODO: Compare with existing codes
% - EFIT (if timing data available from literature)
% - FREEGS (Python)
% - TORAX (JAX-based)
% - Table: code, language, forward time, inverse time, grid size

\begin{table}[htbp]
\centering
\caption{Performance comparison of equilibrium solvers on a
$65\times65$ grid.}
\label{tab:performance}
\begin{tabular}{@{}lllrr@{}}
\toprule
Code & Language & Approach & Forward [\si{\ms}] & Inverse [\si{\ms}] \\
\midrule
EFIT         & Fortran & Picard + LM   & ---  & ---  \\
FREEGS       & Python  & Picard        & ---  & N/A  \\
TORAX        & JAX     & Transport     & ---  & N/A  \\
\textbf{This work} & \textbf{Rust} & \textbf{Picard+SOR / LM} & \textbf{---} & \textbf{---} \\
\bottomrule
\end{tabular}
\end{table}

\lipsum[14]

%% ============================================================
%%  6. DISCUSSION
%% ============================================================
\section{Discussion}
\label{sec:discussion}

% TODO: Discuss results, limitations, future work
% - Accuracy comparable to EFIT for tested discharges
% - Speed advantage of Rust over Python implementations
% - Current limitations: 2D only, no free-boundary yet, limited coil models
% - Future: GPU acceleration, real-time capable, ITER-relevant geometries
% - Comparison with ML-based approaches (emerging trend)

\lipsum[15-16]

%% ============================================================
%%  7. CONCLUSION
%% ============================================================
\section{Conclusion}
\label{sec:conclusion}

% TODO: Summarise contributions and key findings
% - Open-source, validated equilibrium reconstruction code
% - Rust + PyO3 architecture balances performance and usability
% - Validated against DIII-D discharges
% - Available under permissive license

\lipsum[17]

%% ============================================================
%%  8. CODE AVAILABILITY
%% ============================================================
\section{Code Availability}
\label{sec:code-availability}

% TODO: Fill in actual repository URL and version
The source code for \textsc{SCPN-Fusion-Core} is available at
\url{https://github.com/TODO/scpn-fusion-core} under the MIT license.
The version used in this paper is tagged as \texttt{vX.Y.Z}.
Supplementary data and scripts to reproduce all figures are included
in the repository under \texttt{paper/path\_a/reproduce/}.

%% ============================================================
%%  ACKNOWLEDGEMENTS
%% ============================================================
\section*{Acknowledgements}

% TODO: Acknowledge funding, data providers, collaborators
% - DIII-D data provided under GA collaboration agreement
% - Computational resources

%% ============================================================
%%  REFERENCES
%% ============================================================
\bibliographystyle{elsarticle-harv}
\bibliography{references}

\end{document}
